\documentclass[]{ceurart}

\usepackage{soul}
\usepackage{multicol}
\usepackage[dvipsnames]{xcolor}
\usepackage{caption}
\usepackage{booktabs}
\usepackage{listings}
\usepackage{array}
\usepackage{multirow}
\usepackage[noend]{algorithm2e}
\usepackage{amsmath}
\usepackage{enumitem}

\SetNlSty{bfseries}{\color{black}}{}

\lstset{breaklines=true}

\newcommand\ty[1]{\textsf{#1}}
\newcommand\sym[1]{\textsf{#1}}
\newcommand\var[1]{\mathit{#1}}

%% TYPESETTING: hacks
\newcommand\medrightarrow{\mathrel{{{\color{black}\relbar}\kern-0.9ex\rlap{\color{white}\ensuremath{\blacksquare}}\kern-0.9ex}\joinrel{\color{black}\rightarrow}}}
\newcommand\medleftarrow{\mathrel{{\color{black}\leftarrow}\kern-0.9ex\rlap{\color{white}\ensuremath{\blacksquare}}\kern-0.9ex\joinrel{{\color{black}\relbar}}}}
\newcommand\medleftrightarrow{\mathrel{\leftarrow\kern-1.685ex\rightarrow}}
\newcommand\Medrightarrow{\mathrel{{{\color{black}\Relbar}\kern-0.9ex\rlap{\color{white}\ensuremath{\blacksquare}}\kern-0.9ex}\joinrel{\color{black}\Rightarrow}}}
\newcommand\Medleftrightarrow{\mathrel{\Leftarrow\kern-1.685ex\Rightarrow}}

\def\negvthinspace{\kern-0.083333em}
\def\vthinspace{\kern+0.083333em}

\newdefinition{definition}{Definition}
\newdefinition{example}[definition]{Example}

\setlist[itemize]{topsep=5pt, itemsep=0pt}
\setlist[enumerate]{topsep=5pt, itemsep=0pt}

% algo2e package
\SetFuncSty{textit}
\SetFuncArgSty{text}
\SetArgSty{text}
\SetDataSty{text}

\DontPrintSemicolon

% some keywords
\SetKwFunction{GenFormulae}{generate\_mono\_formulae}
\SetKw{ST}{such that}

% bounds keywords
\SetKwData{MonoCap}{\textcolor{ourblueviolet}{mono\_cap}}
\SetKwData{MonoMult}{\textcolor{ourblueviolet}{mono\_mult}}
\SetKwData{MonoFloor}{\textcolor{ourblueviolet}{mono\_floor}}
\SetKwData{PolyCap}{\textcolor{ourblueviolet}{nonm\_cap}}
\SetKwData{PolyMult}{\textcolor{ourblueviolet}{nonm\_mult}}
\SetKwData{PolyFloor}{\textcolor{ourblueviolet}{nonm\_floor}}
\SetKwData{MaxMono}{\textcolor{ourblueviolet}{max\_mono\_args}}
\SetKwData{MaxPoly}{\textcolor{ourblueviolet}{max\_nonm\_args}}
\SetKwData{Limit}{\textcolor{ourblueviolet}{total\_new\_formulae}}
\SetKwData{MonoSubstsLimit}{\textcolor{ourblueviolet}{max\_substs}}
\SetKwData{SubstLimit}{\textcolor{ourblueviolet}{substitution\_limit}}
\SetKwData{Loop}{\textcolor{ourblueviolet}{num\_loops}}
\SetKwData{TypeVars}{type\_variables}
\SetKwData{UsedSubst}{\(S\)}

\SetKw{And}{and}\SetKw{True}{true}\SetKw{False}{false}\SetKw{Stop}{stop}

\SetKwFunction{Max}{max}\SetKwFunction{Min}{min}
\SetKwFunction{Domain}{subst\_domain}\SetKwFunction{Len}{len}
\SetKwFunction{SubstGen}{matches}\SetKwFunction{TyVars}{type\_variables}
\SetKwFunction{FMonoStep}{formula\_mono\_step}
\SetKwProg{Fn}{Function}{}{end}\SetKwFunction{FRecurs}{FnRecursive}%
\newcommand{\forcond}{$i=0$ \KwTo $n$}

\definecolor{ourblueviolet}{HTML}{0071BC}
\definecolor{grey}{rgb}{0.8,0.8,0.8}
\definecolor{algoColorKeyword}{named}{grey}
\makeatletter
% Block with a vertical line
\renewcommand{\algocf@Vsline}[1]{%
   \strut\par\nointerlineskip%
   \algocf@bblockcode%
   \algocf@push{\skiprule}%
   \hbox{{\color{algoColorKeyword}\vrule}%
      \vtop{\algocf@push{\skiptext}%
      \vtop{\algocf@addskiptotal\advance\hsize by -\skiplength #1}}%
   }%
   \algocf@pop{\skiprule}%
   \algocf@bblockcode%
}
\makeatother

\begin{document}

\copyrightyear{2024}
\copyrightclause{Copyright for this paper by its authors. Use permitted under Creative Commons License Attribution 4.0 International (CC BY 4.0)}

\conference{Submitted draft} % TODO update

% TODO, at the end check the following
%   - formulas -> formulae
%   - clause -> formula
%   - terms in ``xxx'' -> \emph{xxx}
%   - monomorphization -> monomorphisation
%   - higher-order -> higher order
%   - TH1 with ugly 1
%   - correctly spelled TPTP, Leo-III, Zipperposition, E, Vampire
%   - polymorphic and monomorphic
%   - title of bits of the algorithms
%   - TH1 ugly for height of 1 reasons, can be fixed with case feature
%   - eg or ex -> e.g.
%   - function symbol -> symbol instance (depends on context)
%   - ... -> \dots
%   - comma splices!!!!
%   - add \; at the end of each algorithm line
%   - ``'' -> `'
%   - use \Phi instead of F for formulae
%   - concrete types -> monomorphic types
%   - use in not \in when we are not precisely dealing with set membership
%   - $\Phi$ for problem and $\varphi_1$ etc.
%   - n-uples -> n-tuples
%   - ize -> ise (in cas i forgot any)
%   - remove vertical space before lists (if allowed)
%   - add full stops to end of sentences in itemize
%   - B\"ome -> B\"ohme

\title{Iterative Monomorphisation}

\author[1,2]{Tanguy Bozec}[%
email=tanguy.bozec@ens-paris-saclay.fr,
]
\author[2]{Jasmin Blanchette}[%
email=jasmin.blanchette@ifi.lmu.de,
%url=https://www.tcs.ifi.lmu.de/mitarbeiter/jasmin-blanchette_de.html,
]
\address[1]{ENS Paris-Saclay, Université Paris-Saclay, France}
\address[2]{Institute of Informatics, Ludwig-Maximilians-Universität München, Germany}


\begin{abstract}
Monomorphisation can be used to extend monomorphic provers to support polymorphic logics. We propose an iterative approach, which is necessarily incomplete but which works well in practice. It is implemented in the Zipperposition prover, where it can be used to translate away polymorphism before invoking the monomorphic prover E as a backend. Our evaluation demonstrates that this approach increases Zipperposition's success rate. Moreover, we find that iterative monomorphisation outperforms native implementations of polymorphism.
\end{abstract}

\begin{keywords}
   Polymorphism\sep
   monomorphism\sep
   automatic theorem proving
\end{keywords}

\maketitle

% target page number: 2-2.5
\section{Introduction}

One of the main applications of automatic theorem provers is to provide automation to users of proof assistants. Many proof assistants, such as HOL4 \cite{slind-norrish-2008}, HOL Light \cite{harrison-2009}, and
Isabelle/HOL \cite{nipkow-et-al-2002}, support rank-1 polymorphism, where type quantification is allowed only at the top level of formulae. On the other hand, many automatic provers operate only on monomorphic logics. One approach to close this gap is to extend provers to natively support polymorphism, as has been done for Vampire \cite{bhayat-reger-2020}. This, however, entails a lot of work that needs to be redone for every prover.

The alternative is to translate polymorphic problems to monomorphic problems.
One approach is to encode polymorphism using dedicated function or predicate symbols in a monomorphic logic;
we refer to Blanchette et al.\ \cite[Section~9]{mono-trans} for an overview of this approach.
Another approach is to encode polymorphism using {iterative monomorphisation}, as described by B\"ohme \cite[Section 2.2.1]{sb-phd}. This method relies on heuristically instantiating the formulae's type variables with concrete types.

By a type version of the compactness theorem, we have that in first order logic, given a polymorphic formula \(\varphi\), there exist an equisatisfiable finite set of monomorphic instances of \(\varphi\). However, such a set cannot be computed \cite[Theorem 1]{expr-poly-types}. As a result, any monomorphisation method based on instantiation of type variables is
bound to be incomplete.

B\"ohme's iterative approach is implemented as part of the SMT (satisfiability modulo theories) integration \cite[Chapter 2]{sb-phd} in Isabelle/HOL. This implementation is also used by Sledgehammer \cite{judgement, hammer} to interface with superposition based automatic theorem provers. However, it is is documented only as a single subsection in his PhD thesis \cite[Section 2.2.1]{sb-phd}.
A similar algorithm appears to be implemented in the \texttt{MESON} tactic \cite{harrison-1996} of HOL Light, but it is undocumented.

In this paper, we present an algorithm based on our understanding of Böhme's description and implementation (Section~\ref{sec:high level-algorithm}). We also provide a more detailed description to help future implementers. In addition, this description shows some of the ways in which an implementation can be made more efficient and avoid explosions (Section~\ref{sec:low-level-algorithm}).

The algorithm works as follows. We assume problems to be sets of formulae. All symbols of the problem are collected, and the polymorphic instances of symbols are matched against the monomorphic ones. This yields new instances of symbols, both polymorphic and monomorphic. The process is then iterated
a number of times, making use of the newly generated instances.

Consider the unary type constructor \ty{list}. If a formula contains $\ty{list}(\alpha)$, where $\alpha$ is a type variable, it may be possible to generate the types $\ty{list}(\ty{int})$, $\ty{list}(\ty{list}(\ty{int}))$, etc. However, because new types emerge through matching, $\ty{list}(\ty{list}(\ty{int}))$ can be obtained only once the $\ty{list}(\ty{int})$ instance has already been generated. This example also makes it clear that the set of instances that can be generated is infinite.

To keep the number of generated formulae finite, we limit the number of iterations. After the iterations are completed, the new monomorphic symbol instances are used to instantiate the polymorphic symbols in the problem's formulae, generating new monomorphic formulae. Finally, because monomorphic provers generally do not support $n$-ary type constructors, types must be `mangled'; for example, the compound type $\ty{list}(\ty{int})$ could be mangled to the constant $\ty{list\_int}$.

We implemented iterative monomorphisation in Zipperposition \cite{zipp}, a higher order prover written in OCaml. Although Zipperposition is polymorphic, it uses the monomorphic prover E \cite{e} as a backend. This means that E can now be used with polymorphic problems. Moreover, our implementation of the algorithm in Zipperposition can be used as a preprocessor to interface with other stand-alone provers.

Our empirical evaluation on TPTP \cite{tptp} problems attempts to answer three questions (Section~\ref{sec:evaluation}):
\begin{enumerate}
\item Is the new Zipperposition with the E backend more successful on polymorphic problems than Zipperposition without backend?

\item How competitive are monomorphic provers on monomorphised polymorphic problems?

\item Is iterative monomorphisation more effective than the native polymorphism implemented in polymorphic provers?
\end{enumerate}

Our findings are as follows:
\begin{enumerate}
\item Zipperposition benefits substantially from the E backend.

\item E with monomorphisation comes close second to the polymorphic prover Vampire.

\item For Leo-III \cite{leo-iii} and Vampire \cite{vamp}, we find that monomorphisation is indeed more effective than native polymorphism.
\end{enumerate}

% target page number: 0.5-1.5
\section{Preliminaries}
\label{sec:preliminaries}

This algorithm works independently of the structure of the problem's formulae. It relies exclusively on the formulae's monomorphic and polymorphic symbol instances. Type variables are assumed to be implicitly universally quantified at a formula's top level. The precise form of formulae is left unspecified.
Due to this generality, iterative monomorphisation can be used with any reasonable rank-1 polymorphic logic. In particular, it can operate in the polymorphic first and higher order logics embodied by TPTP's TF1 and TH1 syntaxes \cite{blanchette-paskevich-2013,th1}, implemented by several automatic provers.


Our abstract framework relies on the following basic definitions.

\begin{definition}
A (\emph{polymorphic}) \emph{type} \(\tau\) can be a type variable (e.g.\ \(\alpha\)) or
the application of an \(n\)-ary type constructor to \(n\) types (e.g.\ \(\ty{list}(\alpha)\), \(\ty{map}(\ty{int},\ty{string})\)).
If $n = 0$, we omit the parentheses (e.g.\ \(\ty{int}\)).
\end{definition}

\begin{definition}
A type is \emph{monomorphic} if it contains no type variables.
\end{definition}

\begin{definition}
A (function or predicate) {symbol} \(f\) has a \emph{type arity} that defines the number of type arguments it takes. A \emph{symbol instance} is a symbol applied to type arguments listed between angle brackets: \(f\langle \tau_1, \dots, \tau_n\rangle\), where each $\tau_i$ is a type. If $n = 0$, we omit the brackets (e.g.\ $f$).
\end{definition}

\begin{definition}
A (\emph{type}) \emph{substitution} is a partial function mapping a finite number of type variables to corresponding types. Substitutions are written as
$\sigma = \{\alpha_1\mapsto\tau_1, \dots, \alpha_n\mapsto\tau_n\}$. They are assumed to be lifted to formulae; thus, $\sigma(\varphi)$ yields the variant of $\varphi$ in which each $\alpha_i$ is replaced by $\tau_i$.
Given two substitutions \(\tau, \upsilon\), the successive application of \(\tau\) and \(\upsilon\) is denoted by \(\upsilon \circ \tau\).
\end{definition}

\begin{definition}
Two substitutions \(\{\alpha_1 \mapsto \tau_1, \dots, \alpha_m\mapsto\tau_m\}\) and \(\{\beta_1 \mapsto \upsilon_1, \dots, \beta_n\mapsto\upsilon_n\}\) are \emph{compatible} if \(\alpha_i = \beta_j\) implies \(\tau_i = \upsilon_j\) for all \(i, j\).
\end{definition}


\begin{definition}
Given two types \(\tau, \upsilon\), \emph{matching} \(\upsilon\) against \(\tau\) will either fail or yield a substitution \(\sigma\) such that \(\sigma(\upsilon) = \tau\).
\end{definition}

In the following sections, we will always match a polymorphic type $\upsilon$ against a monomorphic type $\tau$.

% target page number: 2-4
\section{High level algorithm}
\label{sec:high level-algorithm}

The iterative monomorphisation algorithm takes a polymorphic problem as input and returns a monomorphic problem. It operates by applying an arbitrary number of iterations. Each iteration takes a polymorphic problem as argument and returns a problem with additional partially instantiated formulae. A single iteration consists of a collection phase and an instantiation phase. Once the iterations are completed, a final step filters out all non-monomorphic formulae returned by the last iteration.

The initial phase of each iteration consists of computing two maps, \(M\) and \(N\), from the input problem~$\Phi$.
%
\begin{enumerate}
\item[\labelitemi] Given a symbol \(f\) occurring in \(\Phi\), the set \(M(f)\) consists of all monomorphic type arguments tuples to which \(f\) is applied in \(\Phi\). For example, if \(\sym{foldl}\langle \ty{nat}, \ty{int}\rangle\) occurs in \(\Phi\), then \((\ty{nat}, \ty{int}) \in M(\sym{foldl}) \).

\item[\labelitemi] Given a formula \(\varphi \in \Phi\) and a symbol \(f\) occurring in \(\varphi\), the set \(N(\varphi)(f)\) consists of all type arguments tuples to which \(f\) is applied in \(\varphi\) and which contains a type variable. For example, if \(\sym{foldl}\langle \ty{nat}, \ty{list}(\alpha)\rangle\) occurs in \(\varphi\), then \((\ty{nat}, \ty{list}(\alpha)) \in N(\varphi)(\sym{foldl}) \).
\end{enumerate}

These definitions depend on the precise form of the input formulae, which has been left undefined. It is important to parametrise \(N\) with \(\varphi\) because type variables are implicitly quantified at the formula level. The formula indicates the scope of type variables. This is not necessary for \(M\) since all the types it contains are monomorphic.

Once the maps \(M\) and \(N\) are initialised, each iteration performs the following steps to create new instances of formulae:

\begin{enumerate}

   \item Create an empty set of formulae \(\Phi'\).

   \item For each formula \(\varphi \in \Phi\) and for each symbol \(f\) occurring in \(\varphi\):
   \begin{enumerate}
    \item[2.1.] For each tuple \((\tau_1, \dots, \tau_n) \in  M(f)\) and each tuple \((\upsilon_1, \dots, \upsilon_n) \in N(\varphi)(f)\),
     for each \(i\), match \(\upsilon_i\) against \(\tau_i\), yielding the substitution \(\sigma_i\) in case of success.

    \item[2.2.] If all \(n\) matchings are successful and the substitutions \(\sigma_i\) are pairwise compatible,
add the formula \((\sigma_1 \circ \dots \circ \sigma_n)(\varphi)\) to \(\Phi'\).
   \end{enumerate}

   \item Return \(\Phi \cup \Phi'\).

\end{enumerate}

The algorithm is clearly sound because the newly generated formulae are instances of the initial problem's formulae, where type variables have been instantiated with monomorphic types. It is, however, not guaranteed to be complete.

\begin{example}Consider the following problem:
\begin{enumerate}
   \item \(\sym{p}\langle \ty{int}\rangle(0)\)
   \item \(\forall a: \alpha{,}\; \mathit{as}:\ty{list}(\alpha){,}\; \sym{p}\langle\alpha\rangle(a) \rightarrow \sym{p}\langle \ty{list}(\alpha)\rangle(\mathit{as})\)
\end{enumerate}
%
The first iteration matches \(\alpha\) against \(\ty{int}\) for $\sym{p}$, generating the formula
%
\begin{enumerate}
   \item[3.] \(\forall a: \ty{int}{,}\; \mathit{as}:\ty{list}(\ty{int}){,}\; \sym{p}\langle\ty{int}\rangle(a) \rightarrow \sym{p}\langle \ty{list}(\ty{int})\rangle(\mathit{as})\)
\end{enumerate}
%
The second iteration matches \(\alpha\) against \(\ty{list}(\ty{int})\), leading to the formula
%
\begin{enumerate}
   \item[4.] \(\forall a: \ty{list}(\ty{int}){,}\; \mathit{as}:\ty{list}(\ty{list}(\ty{int})){,}\; \sym{p}\langle\ty{list}(\ty{int})\rangle(a) \rightarrow \sym{p}\langle \ty{list}(\ty{list}(\ty{int}))\rangle(\mathit{as})\)
\end{enumerate}
%
Similarly the third iteration adds
%
\begin{enumerate}
   \item[5.] \(\forall a: \ty{list}(\ty{list}(\ty{int})){,}\; \mathit{as}:\ty{list}(\ty{list}(\ty{list}(\ty{int}))){,}\; \sym{p}\langle\ty{list}(\ty{list}(\ty{int}))\rangle(a) \rightarrow \sym{p}\langle \ty{list}(\ty{list}(\ty{list}(\ty{int})))\rangle(\mathit{as})\)
\end{enumerate}

\end{example}

This example illustrates how an infinite number of new formulae can be generated from a simple initial problem. Although the example may seem artificial [contrived?], similar problems frequently arise in practice. For example, the \sym{concat} function of Isabelle/HOL, which is characterised in the base case by the equation \(\sym{concat}\langle\alpha\rangle\; \sym{Nil}\langle\ty{list}(\alpha)\rangle = \sym{Nil}\langle\alpha\rangle\), exhibits the same behavior. Any reasonable implementation requires bounds limiting the number of new type arguments, substitutions and formulae to be useful.

% target page number: 2-4
\section{Low level algorithm}
\label{sec:low-level-algorithm}

The algorithm presented in the previous section is simple but too naive for an implementation. In this section, we present a lower level algorithm with the following features.
First, numeric bounds are introduced to stop explosive enumerations.
Second, type arguments tuples are separated into an old set and a new set to avoid re-computing some of the same matchings in successive iterations.
Third, substitutions are directly applied to the type arguments instead of the formulae. This dispenses from having to re-extract the type arguments from the formulae at each iteration. New formulae are generated only once all iterations are complete, in a separate, final step.

The data structures used in the pseudocode are based on the ones used in the high level description. Instead of a map \(M\) from symbols to monomorphic type argument tuples, we now have \(M_\text{old}\) and \(M_\text{new}\), which play the same role whilst also distinguishing between type arguments tuples having already been matched against and those that have not. Similarly, \(N_\text{old}\) and \(N_\text{new}\) replace the map \(N\) from formulae to symbols to non-monomorphic type argument tuples. Finally, we keep track of a set \(S\) of substitutions generated by the matchings. It is used to generate new formulae in the final phase.

All sets referenced in the pseudocode are assumed to be finite. Additionally, this algorithm relies on primitives whose implementation depends on the specifics of the grammar and logic used. They will therefore not be expanded upon. Functions computing the following are assumed to be available: 
\begin{enumerate}
   \item[\labelitemi] \emph{initialisation\((\Phi)\)}, where \(\Phi\) is a set of (polymorphic) formulae, extracts the initial type argument maps \(M\) and \(N\) from \(\Phi\).
   \item[\labelitemi] \emph{type\_variables\((\tau_1, \dots, \tau_n)\)}, where \(\tau_1, \dots,\tau_n\) are types, gathers all the type variables from each type \(\tau_1, \dots, \tau_n\) into a set.
   \item[\labelitemi] \emph{match\((\upsilon, \tau)\)}, where \(\upsilon\) and \(\tau\) are types, will match \(\upsilon\) against \(\tau\) and either fail or return \emph{Success\((\sigma)\)}, where \(\sigma\) is the substitution resulting from the matching. The pseudocode matches only non-monomorphic types against monomorphic types.
   \item[\labelitemi] \emph{domain\((\sigma)\)}, where \(\sigma\) is a substitution, returns the set of type variables \(\alpha\) such that \(\sigma(
   \alpha) \not= \alpha\).
   \item[\labelitemi] \emph{compatible\((\sigma_1, \sigma_2)\)}, where \(\sigma_1\) and \(\sigma_2\) are substitutions, tests the compatibility between \(\sigma_1\) and \(\sigma_2\).
   \item[\labelitemi] Composition and application of substitutions are written using mathematical notations.
   \item[\labelitemi] \emph{mangle\((\Phi)\)}, where \(\Phi\) is a set of monomorphic formulae returns the same set of formulae where all types have been mangled.
\end{enumerate}

\SetKwFunction{IterMono}{iterative\_monomorphisation}
\begin{figure}[b]
\begin{quote}
\begin{algorithm}[H]
\Fn(){\(\IterMono(\Phi)\)}{
   \SetKw{And}{and}\SetKw{True}{true}\SetKw{False}{false}\SetKw{To}{to}

   \SetKwData{Problem}{\(\Phi\)}\SetKwData{AllSubst}{\(S\)}

   \SetKwFunction{Init}{initialisation}\SetKwFunction{FStep}{formula\_mono\_step} 
   \SetKwFunction{TyVars}{type\_variables}\SetKwFunction{MonoSubst}{mono\_substs}
   \SetKwFunction{GenFormulae}{generate\_mono\_formulae}
   \SetKwFunction{Mangle}{mangle}

   \KwData{set \(\Phi\) of polymorphic formulae}
   \KwResult{set of monomorphic formulae}

   \BlankLine

   \((M_{\text{old}}, N_{\text{old}}) \leftarrow (\emptyset, \emptyset)\)\;
   \((M_{\text{new}}, N_{\text{new}}) \leftarrow \Init(\Problem)\)\;

   \AllSubst\(\leftarrow \emptyset\)\;

   \BlankLine

   \For{\(i = 1\) \To \textcolor{ourblueviolet}{\Loop}}{
      \((M_{\text{next}}, N_{\text{next}})\leftarrow(\emptyset,\emptyset)\)\;
      \ForEach{\(\varphi \in \Problem\)} {
         \((M_{\Delta}, N_{\Delta}(\varphi), S_{\Delta})\leftarrow \FStep(M_{\text{old}}, N_{\text{old}}(\varphi), M_{\text{new}}, N_{\text{new}}(\varphi))\)\;
         \(\AllSubst\leftarrow\AllSubst\cup S_{\Delta}\)\;
         \(M_{\text{next}}\leftarrow M_{\text{next}}\cup M_{\Delta}\)\;
         \(N_{\text{next}}(\varphi)\leftarrow N_{\text{next}}(\varphi)\cup N_{\Delta}(\varphi)\)\;
      }

      \((M_{\text{old}}, M_{\text{new}})\leftarrow (M_{\text{old}}\cup M_{\text{new}}, M_{\text{next}})\)\;
      \((N_{\text{old}}, N_{\text{new}})\leftarrow (N_{\text{old}}\cup N_{\text{new}}, N_{\text{next}})\)\;

   }

   \BlankLine

   \Return \(\Mangle(\GenFormulae(\Phi, S))\)
}
\end{algorithm}
\end{quote}
\caption{Pseudocode for iterative monomorphisation}
\label{iter_mono}
\end{figure}


The iterative monomorphisation algorithm is given in Figure~\ref{iter_mono}. It has three phases. The first phase applies a \emph{monomorphisation step} to each formula in \(\Phi\) until the user-set limit, \textcolor{ourblueviolet}{num\_loops}, is reached. This limit is the only bound necessary for the algorithm to terminate. We use the colour blue to identify bounds and code related to bounds. At the end of each of these iterations, the old and new type argument maps are updated with newly generated types. Once these iterations are completed, the first phase is done and the substitutions used to create new type arguments tuples are passed to $\mathit{generate\_mono\_formulae}$ for the second phase. The third phase mangles the composite types of the newly monomorphised formulae. This allows targeting a simply typed logic with no general support for $n$-ary type constructors.
% \cite[Sections 4 and 5]{mono-trans}.
% Paper name is gt

\begin{figure}
\begin{quote}
\begin{algorithm}[H]

   \Fn(){\(\FMonoStep(\varphi, M_{\text{old}}, M_{\text{new}}, N_{\text{old}}(\varphi), N_{\text{new}}(\varphi)\)}{

   \KwData{\begin{minipage}[t]{.8\textwidth}
     \strut polymorphic formula $\varphi$ \\
     old and new monomorphic type argument maps \(M_{\text{old}}, M_{\text{new}}\) \\
     old and new non-monomorphic type argument maps \(N_{\text{old}}(\varphi), N_{\text{new}}(\varphi)\)\strut
      \end{minipage}
   }
   \KwResult{\begin{minipage}[t]{.8\textwidth}
      \strut monomorphic type argument map \\
      non-monomorphic type argument map \\
      set of substitutions\strut
      \end{minipage}
   }
   \BlankLine

   \(\TypeVars \leftarrow \TyVars(\varphi)\)\;

   \textcolor{ourblueviolet}{
   \(\MaxMono \leftarrow \Max(\Min(\MonoFloor, |M_{\text{old}}\cup M_{\text{new}}| \cdot \MonoMult), \MonoCap)\)\;
   \(\MaxPoly \leftarrow \Max(\Min(\PolyFloor, |N_{\text{old}}(\varphi)\cup N_{\text{new}}(\varphi)| \cdot \PolyMult), \PolyCap)\)\;
   }

   \BlankLine

   \UsedSubst \(\leftarrow\emptyset\)\;

   \BlankLine

   \ForEach{\(\sigma\in \SubstGen(M_{\text{new}}, N_{\text{new}}(\varphi))
   \cup \SubstGen(M_{\text{new}}, N_{\text{old}}(\varphi))
   \cup \SubstGen(M_{\text{old}}, N_{\text{new}}(\varphi))\)}{
   \ForEach{\(f\mapsto (\upsilon_1, \dots,\upsilon_n)\in N_{\text{old}}(\varphi)\cup N_{\text{new}}(\varphi)\)}{
            \eIf{\(\TyVars(\upsilon_1, \dots, \upsilon_n) \subseteq \Domain(\sigma)\)}{
               \textcolor{ourblueviolet}{
               \uIf{\(|M_{\text{next}}| < \MaxMono\)}{
                  \textcolor{black}{
                  \(M_{\text{next}}(f)\leftarrow M_{\text{next}}(f)\cup\{(\sigma(\upsilon_1),\dots,\sigma(\upsilon_n))\}\)\;
                  \UsedSubst \(\leftarrow \UsedSubst \cup\{\sigma\}\)\;}
               }
               \uElseIf{\(|N_{\text{next}}(\varphi)| \geq \MaxPoly\) }{
                  \(M_{\text{next}} \leftarrow M_{\text{next}}\setminus (M_{\text{old}} \cup M_{\text{new}})\)\;
                  \(N_{\text{next}}(\varphi)\leftarrow N_{\text{next}}(\varphi)\setminus (N_{\text{old}}(\varphi)\cup N_{\text{new}}(\varphi))\)\;
                  \Return \((M_{\text{next}}, N_{\text{next}}(\varphi), \UsedSubst)\)
               }}
            }{\textcolor{ourblueviolet}{
               \If{\(|N_{\text{next}}(\varphi)| < \MaxPoly\)}{ \textcolor{black} {
               \(N_{\text{next}}(\varphi)(f)\leftarrow N_{\text{next}}(\varphi)(f)\cup\{(\sigma(\upsilon_1),\dots,\sigma(\upsilon_n))\}\)\;
               \(\UsedSubst \leftarrow \UsedSubst \cup\{\sigma\}\)\;
               }}
            }}
      }
   }

   \BlankLine

   \Return \((M_{\text{next}}\setminus (M_{\text{old}} \cup M_{\text{new}}), N_{\text{next}}(\varphi)\setminus (N_{\text{old}}(\varphi)\cup N_{\text{new}}(\varphi)), \UsedSubst)\)
}

\end{algorithm}
\end{quote}
\caption{Pseudocode for formula monomorphisation step}
\label{mono_step}
\end{figure}


The formula monomorphisation algorithm is given in Figure~\ref{mono_step}. It forms the core of the monomorphisation process by computing new type arguments tuples for a single formula. Type argument tuples are matched against each other to obtain a set of substitutions which is iterated over in the outermost loop. 
The separation of type arguments tuples into old and new maps is put to use to ensure that only combinations involving at least one new map are considered. This avoids re-computing some of the matchings processed in previous iterations. Not all such redundant computations can be avoided because the same substitution can be obtained from different matchings. New type argument tuples are obtained by applying each substitution to each non-monomorphic type argument tuple such that at least one type variable in the tuple is instantiated by the substitution.


The total number of type argument tuples can increase cubicly in the number of type arguments tuples at each iteration and can therefore grow doubly exponentially in the number of iterations. We sketch an example of such growth, for a single formula. If we assume that after \(k\) iterations, there are \(N_k\) total type argument tuples divided evenly between monomorphic and non-monomorphic type arguments tuples, then there can be up to \(\bigl(\frac{N_k}{2}\bigr)^2\) successful matches, yielding as many substitutions. Each substitution is then applied to each non-monomorphic type argument for a total of \(\bigl(\frac{N_k}{2}\bigr)^3\) possible new type argument tuples. If half of these new type argument tuples are monomorphic and the other half are non-monomorphic, then the next iteration will begin with \(N_{k+1} = N_k^3 \cdot 2^{-3}\) evenly split type argument tuples. Therefore, the total number of type arguments tuples on the \(k\)th iteration can reach \(N_k = N_0^{3^k} \cdot 2^{\frac{-3^{k+2}+3}{2}}\).

   Depending on the shape and size of the input problem and the number of iterations performed, the doubly exponential growth may be problematic. Introducing bounds addresses this potential issue. The limit on the number of new monomorphic type argument tuples is \(\min(\max(\MonoMult\cdot m, \MonoFloor), \MonoCap)\), where \(m\) is the total number of monomorphic type argument tuples. The components of this limit are
\begin{enumerate}
   \item \MonoCap, an absolute limit on the total number of new type argument tuples;
   \item \MonoMult, which is used to allow the total number of (monomorphic) type argument tuples to grow by a certain proportion of the current number \(m\) of monomorphic type argument tuples;
   \item \MonoFloor, which balances out \MonoMult, preventing \MonoMult from inhibiting new type argument tuple generation if \(m\) is too low.
\end{enumerate}

Similar bounds are used for the non-monomorphic type arguments tuples:
The limit on the number of new non-monomorphic type argument tuples is \(\min(\max(\PolyMult\cdot n, \PolyFloor),\allowbreak \PolyCap)\), where \(n\) is the number of non-monomorphic type argument tuples associated with the current formula. A significant difference with the monomorphic case is that \(n\) depends on the current formula being processed whilst \(m\) does not.
Both in the monomorphic and in the non-monomorphic case, the maximum number of newly generated type argument tuples is fixed per formula and per iteration.

The \emph{matches} function, which computes the substitutions used for generating new type arguments, is given in Figure~\ref{subst_gen}. Each symbol instance from \(N(\varphi)\) is matched against all corresponding symbol instances from \(M\).
For two such symbol instances, types from the non-monomorphic type argument tuple are matched component-wise against the types from the monomorphic type argument tuple. The resulting substitutions are composed if they are compatible. In the pseudocode, this is checked by making sure the \textbf{while} loop has successfully iterated over all elements of the type argument tuple. If any substitutions are incompatible, the matchings are discarded. Since the composition of two compatible substitutions is commutative, the order of composition is irrelevant. The total number of substitutions generated is capped by \textcolor{ourblueviolet}{substitution\_limit}.

\begin{figure}
\begin{quote}
\begin{algorithm}[H]
\SetKwFunction{SubstGen}{matches}
\Fn(){\(\SubstGen(M, N(\varphi))\)}{
   \SetKw{And}{and}\SetKw{True}{true}\SetKw{False}{false}
   \SetKwFunction{Success}{Success}\SetKwFunction{Compatible}{compatible}
   \SetKwFunction{Match}{match}
   \SetKwInOut{Input}{input}\SetKwInOut{Output}{output}
   \SetKw{Break}{break}

   \KwData{\begin{minipage}[t]{.8\textwidth}
      \strut monomorphic type argument map \(M\) \\
      non-monomorphic type argument map \(N(\varphi)\)\strut
      \end{minipage}
   }
   \KwResult{set of substitutions}
   \BlankLine

   \(S\leftarrow \emptyset\)\;

   \BlankLine

   \ForEach{\(f\mapsto (\upsilon_1, \dots,\upsilon_n)\in N(\varphi)\)}{
      \ForEach{\( (\tau_1, \dots,\tau_n) \in M(f) \)}{
         \(\sigma\leftarrow \{\}\)\;
         \(i \leftarrow 1\)\;
         \While{\(i \leq n\)}{
            \eIf{\(\Match(\upsilon_i, \tau_i)\) has the form \(\Success(\sigma_i)\) \And \(\Compatible(\sigma, \sigma_i)\)}{
               \(\strut\sigma\leftarrow\sigma_i\circ\sigma\)
            }
            {\Break}

            \(i\leftarrow i+1\)\;
         }
         
         \If{\(i > n\)}{
            \textcolor{ourblueviolet}{
               \eIf{\(|S| < \SubstLimit\)}{
                  \textcolor{black}{\(S\leftarrow S\cup \{\sigma\}\)\;}
               }
               {\Return \(S\)}
            }
         }
      }

   }

   \BlankLine

   \Return \(S\)\;

}
\end{algorithm}
\end{quote}
\caption{Pseudocode for match generation}
\label{subst_gen}
\end{figure}

\begin{figure}
\begin{quote}
\begin{algorithm}[H]
\Fn(){\(\GenFormulae(\Phi, S)\)}{
   \SetKw{And}{and}\SetKw{True}{true}\SetKw{False}{false}\SetKw{To}{to}

   \SetKwData{NewFormulae}{\(\Psi\)}
   \SetKwData{Problem}{\(\Phi\)}\SetKwData{AllSubst}{\(S\)}
   \SetKwData{Limit}{total\_new\_formulae}

   \SetKwFunction{Init}{initialisation}\SetKwFunction{FStep}{formula\_mono\_step} 
   \SetKwFunction{TyVars}{type\_variables}\SetKwFunction{MonoSubst}{mono\_substs}
   \SetKwFunction{GenFormulae}{generate\_formulae}

   \KwData{\begin{minipage}[t]{.8\textwidth}
     \strut set \(\Phi\) of polymorphic formulae \\
     set $S$ of substitutions\strut
     \end{minipage}}
   \KwResult{set of monomorphic formulae}

   \BlankLine

   \NewFormulae \(\leftarrow\emptyset\)\;

   \BlankLine

   \ForEach{\(\varphi\in\Phi\) \ST $\varphi$ is non-monomorphic}{
      \ForEach{\(\sigma\in \MonoSubst(\AllSubst, \TyVars{\(\varphi\)}, \emptyset, \{\})\)}{
         \textcolor{ourblueviolet}{
         \eIf{\(|\NewFormulae|<\Limit\)}{
            \textcolor{black}{
            \(\NewFormulae\leftarrow\NewFormulae\cup \{\sigma(\varphi)\}\)\;}
         }
         {\Return \NewFormulae}
      }}
   }

   \BlankLine

   \Return \NewFormulae
}
\end{algorithm}
\end{quote}
\caption{Pseudocode for monomorphic formula generation}
\label{gen_formulae}
\end{figure}


The various bounds presented here overlap to some extent. For instance, having at most \textcolor{ourblueviolet}{substitution\_limit} substitutions generated by \emph{matches} may be sufficient to curb the number of new type argument tuples, making the \textcolor{ourblueviolet}{\MonoCap}, \textcolor{ourblueviolet}{\MonoMult}, \textcolor{ourblueviolet}{\MonoFloor} triplet superfluous. Each bound nonetheless has its uses. For example, problems that lead to few successful matches but many type arguments tuples may benefit from a limit on the number of new type arguments tuples whilst problems for which substitution generation is more explosive may benefit from a limit on the number of generated substitution.

\begin{figure}
\begin{quote}
   \begin{algorithm}[H]

   \SetKwFunction{Compatible}{compatible}
   \SetKwFunction{MonoSubst}{mono\_substs}
   \SetKwFunction{Domain}{domain}
   \SetKw{And}{and}\SetKw{True}{true}\SetKw{False}{false}\SetKw{Stop}{stop}
   \SetKw{Let}{let}
   \SetKwInOut{Input}{input}\SetKwInOut{Output}{output}

   \Fn(){\(\MonoSubst(S, V, S_{\text{res}}, \sigma)\)}{

   \KwData{\begin{minipage}[t]{.8\textwidth}
     \strut set \(S\) of substitutions \\
     set \(V\) of type variables \\
     \strut set \(S_{\text{res}}\) of substitutions \\
     substitution \(\sigma\)\strut
      \end{minipage}
   }

   \KwResult{set of substitutions}
   \BlankLine

   \eIf{\(V = \emptyset\)}{
    \Return \(S_{\text{res}}\cup\{\sigma\}\)\;%}
   }
   {
      \Let \(\alpha \) \ST \(\alpha\in V\)\;
      \ForEach{\(\sigma_{\negvthinspace\Delta}\in S\) \ST \(\alpha\in\Domain(\sigma_{\negvthinspace\Delta})\) \And \(\Compatible(\sigma,\sigma_{\negvthinspace\Delta})\)}{
         \textcolor{ourblueviolet}{
         \eIf{\(|S_{\text{res}}|<\MonoSubstsLimit\)}{
            \textcolor{black}{
            \(S_{\text{res}}\leftarrow \MonoSubst(S, V\setminus\Domain(\sigma_{\negvthinspace\Delta}), S_{\text{res}}, \sigma_{\negvthinspace\Delta}\circ\sigma)\)\;}
         }
         {\Return \(S_{\text{res}}\)}}
      }
      \Return \(S_{\text{res}}\)\;
   }
   }


\end{algorithm}
\end{quote}
\caption{Pseudocode for monomorphising substitution generation}
\label{mono_substs}
\end{figure}

Once all monomorphisation iterations have been completed, we are left with a set of the substitutions that have been used to generate new type arguments. The last phase uses these substitutions to instantiate the type variables in the input problem's non-monomrophic formuale. In the presence of bounds, the order in which the elements of the set \(S\) of substitutions is traversed affects the formulae resulting from the last phase. As we will discuss in Section~\ref{param_opti}, the best results were obtained when substitutions generated in the same iteration are grouped together.

The \emph{generate\_mono\_formulae} function is given in Figure~\ref{gen_formulae}.
It generates monomorphising substitutions and applies them to the polymorphic formula of the input problem that they instantiate. Since the substitutions are monomorphising relative to the formula they applied to, the resulting formulae are monomorphic. The \textcolor{ourblueviolet}{total\_new\_formulae} bound is used to control the total number of new formulae. It overlaps with \textcolor{ourblueviolet}{max\_substs} but can be useful to set an absolute limit on the size of the final problem.
%In the Zipperposition implementation, this bound is one of the more important ones as the performance of the E prover can be significantly affected by the number of formuale it is given.

To monomorphise a polymorphic formula, we first compute its {monomorphising substitutions} using the \emph{mono\_substs} function given in Figure~\ref{mono_substs}.
A substitution \(\sigma\) is \emph{monomorphising} for a formula \(\varphi\) if \(\sigma(\varphi)\) is monomorphic. Such substitutions are computed using a recursive function. Given a set \(V\) of type variables and a set \(S\) of substitutions, it will select a substitution \(\sigma_{\negvthinspace\Delta}\) from \(S\) that instantiates at least one of the type variables in \(V\). It is important that \(\sigma_{\negvthinspace\Delta}\) be compatible with \(\sigma\) so that they can be composed and the function recursively called to instantiate the remaining type variables.

The Zipperposition implementation of the \emph{mono\_substs} function uses a map from type variables to substitutions instead of a set to filter the relevant substitutions from \(S\) more efficiently. The \textcolor{ourblueviolet}{max\_substs} bound exists for two main reasons:
\begin{enumerate}
   \item The iterative monomorphisation algorithm can generate up to \textcolor{ourblueviolet}{max\_substs} new monomorphic formula per initial polymorphic formula. Generating an excessive number of new formulae can flood the downstream prover. The final number of output formulae is ensured to be at most \(|\Phi|\cdot \text{\textcolor{ourblueviolet}{max\_substs}}\).
   \item The \emph{mono\_substs} function is the most explosive part of the algorithm. If \(S\) contains \(n\) substitutions that each instantiate exactly one of \(v\) different type variables, up to \(n^v\) monomorphising substititions may be generated. Recall that the total number of type argument tuples that are used to generate \(S\) can be doubly exponential in the number of loop iterations.
\end{enumerate}
%[This function can be thought of as exploring the tree of all possible monomorphising substitutions, \(S_{\text{res}}\) would represent the leaves that have already been visited and \(\sigma\) would represent the current node of the tree.]

% target page number: 2-4
\section{Evaluation}
\label{sec:evaluation}

The monomorphisation algorithm is parametrised by many bounds. The first part of the evaluation process is to find appropriate values for these bounds. The second part compares the performance of Zipperposition with and without monomorphisation. The third part compares the performance of different provers on polymorphic problems and their monomorphised counterparts.

The benchmarks are taken from the TPTP library \cite{tptp}. The TPTP library, in version 8.2.0, contains 1765 problems in the TF1 and TH1 grammars, corresponding respectively to first-order and higher-order logic with rank-1 polymorphism. Because Zipperposition does not support reasoning with real numbers, we removed all problems that include real numbers. In total, our benchmark suite contains 1534 polymorphic problems. We chose as a measure of success for a given prover (or prover configuration) the number of problems that could each be solved by the prover in under 30 seconds with a single thread. This corresponds to the default time limit in Sledgehammer.

\subsection{Parameter optimisation}
\label{param_opti}

Each bound of the monomorphisation process presents a tradeoff: a higher bound allows for a more exhaustive instantiations of type variables at the cost of taking more time. Ideally, all possible combinations of values for all bounds could be exhaustively tested to find the best compromise between completeness and speed. Given the number of bounds, however, this is not feasible. Instead, we group closely related bounds together and test combinations of values for bounds in these groups. Once we find the best performing set of values for a group of bounds, we assign these values to the corresponding bounds as we begin the search for the next group. If several groups of bounds result in the same number of proved problems, the most constraining bound values are selected, they are shown in bold in Tables \ref{mono_ty_args} to \ref{mono_time}.

As a precaution against overfitting, we carried out the part of the evaluation related to parameter optimisation and all preliminary evaluations on 500 randomly chosen problems out of the 1534 selected problems. We carried out the rest of the evaluation on the remaining 1034 problems. Before finding values to assign to the bounds of the monomorphisation algorithm, we must make the choice of the base options used to run Zipperposition. Zipperposition has a portfolio mode with several configurations. Because the space of possible base configurations is too large to evaluate, we evaluated, in a preliminary experiment, all portfolio configurations that called E against our benchmark suite of 500 problems, without using monomorphisation. The best performing configuration became the base configuration, which we used as a basis to evaluate the different monomorphisation options. Our preliminary evaluation found that the so-called \verb|40_b.comb| configuration performed best by proving 131 problems.
We conducted additionnal informal preliminary evaluations on the 500 problems benchmark suite to find appropriate default values and test ranges for the monomorphisation bounds. We started the option evaluation process with base configuration \verb|40_b.comb|, with the following default values for the monomorphisation bounds:

\begin{multicols}{2}% 3 columns?
\begin{itemize}
   \item \MonoCap: irrelevant
   \item \MonoMult: irrelevant
   \item \MonoFloor: irrelevant
   \item \PolyCap: \(\infty\)
   \item \PolyMult: \(1\)
   \item \PolyFloor: \(50\)
   \item substitution ordering: separation
   \item \SubstLimit: \(\infty\)
   \item \MonoSubstsLimit: \(10\)
   \item \Limit: \(2000\)
   \item new formulae limit multiplier: \(0\)
   \item E timeout: \(30\)
   \item monomorphisation timeout: \(20\)
   \item \Loop: \(4\)
\end{itemize}
\end{multicols}
Monomorphistion (and the subsequent call to E) occurs at the beginning of the invocation of Zipperposition. Departing from the original \verb|40_b.comb| configuration, which would otherwise call E after 15\% of the total alloted time for Zipperposition has passed. All others options either appear explicitely in the following tables or have been left unmodified.

Table~\ref{mono_ty_args} groups bounds that control the maximum number of newly generated monomorphic type arguments per formula per iteration. The limit on newly generated type arguments is determined by three components that form a natural group of bounds. The table shows that generating no new monomorphic type arguments seems to be the best approach. This result may seem counterintuitive, but it is possible to monomorphise formulae without generating monomorphic type arguments. This is because non-monomorphic type argument generation can generate substitutions that instantiate one or more type variables, and it is these substitutions that are used to monomorphise formulae.

\begin{table}[th]
\caption{Evaluation of bounds for monomorphic type argument generation}
\centering\begin{tabular}{@{}l*{12}{>{\centering\arraybackslash}p{1.5em}}@{}}
   \toprule
   & &&& \multicolumn{6}{c}{cap} \\
   & \multicolumn{4}{c}{500} &\multicolumn{4}{c}{1000} & \multicolumn{4}{c}{\(\infty\)}\\
   \cmidrule(l){2-13}
   & &&& \multicolumn{6}{c}{floor} \\
   \multirow{1}{2em}{mult} & 0 & 50 & 100 & 200& 0 & 50 & 100 & 200& 0 & 50 & 100 & 200\\
    \cmidrule(lr){2-5} \cmidrule(lr){6-9} \cmidrule(l){10-13} 
    0       &\bf{178}& 161 & 161 & 156 & 178 & 160 & 160 & 156 & 178 & 161 & 160 & 156 \\
    1          & 155 & 155 & 155 & 158 & 153 & 154 & 154 & 156 & 154 & 154 & 155 & 155 \\
    2          & 154 & 154 & 153 & 154 & 153 & 153 & 154 & 152 & 154 & 153 & 154 & 154 \\
    \(\infty\) & 153 & 154 & 153 & 155 & 155 & 153 & 154 & 156 & 159 & 160 & 161 & 161 \\
    \bottomrule
\end{tabular}
\label{mono_ty_args}
\end{table}

Table~\ref{nmon_ty_args} plays the same role as Table~\ref{mono_ty_args} except that it evaluates the bounds limiting the number of new \emph{non-}monomorphic type arguments. The bound values are lower because we found non-monomorphic type arguments to be explosive in preliminary evaluations. The table confirms that non-monomorphic type argument generation drives the creation of useful non-monomorphic formulae. This is indicated by the very low number of problems solved when no new non-monomorphic type arguments are allowed. Performance of the monomorphisation algorithm seems to plateau for certain ranges of the tested values and drops off beyond.
%TODO check that 125 is what we get if we just run Zipperposition for 10 seconds (maybe with all additionnal problems that are easily monomorphised)

\begin{table}[th]
\caption{Evaluation of bounds for non-monomorphic type argument generation}
\centering\begin{tabular}{@{}l*{12}{>{\centering\arraybackslash}p{1.5em}}@{}}
   \toprule
   & &&& \multicolumn{6}{c}{cap} \\
   & \multicolumn{4}{c}{500} &\multicolumn{4}{c}{1000} & \multicolumn{4}{c}{\(\infty\)}\\
   \cmidrule(l){2-13}
   & &&& \multicolumn{6}{c}{floor} \\
   \multirow{1}{3em}{mult} & 0 & 10 & 50 & 100& 0 & 10 & 50 & 100& 0 & 10 & 50 & 100\\
    \cmidrule(lr){2-5} \cmidrule(lr){6-9} \cmidrule(l){10-13} 
    0         &125&\bf{184}& 182 & 177 & 125 & 184 & 182 & 177 & 125 & 184 & 182 & 177 \\
    0.5        & 176 & 184 & 182 & 177 & 176 & 184 & 182 & 177 & 176 & 184 & 182 & 177 \\
    1          & 182 & 181 & 178 & 177 & 182 & 181 & 178 & 177 & 182 & 181 & 178 & 177 \\
    \(\infty\) & 173 & 174 & 174 & 174 & 174 & 174 & 173 & 173 & 125 & 125 & 125 & 125 \\
    \bottomrule
\end{tabular}
\label{nmon_ty_args}
\end{table}

%The substitution generation phase occurs once all type arguments have been generated. The bounds limiting the number of new substitutions and the number of monomorphising substitutions per formula are directly related through the role they play in the construction of monomorphising substitutions. The substitution ordering heuristic has a significant impact on monomorphising substitution generation. The ``age'' ordering of substitution orders substitutions generated in earlier iterations are ordered first. The ``random'' ordering randomly shuffles substitutions. Finally, the ``separation'' ordering separates the substitutions into different sets depending on which iteration they were generated in and generates monomorphising substitutions from each of these sets independently. Table \ref{subst_gen_table} shows that the values of bounds limiting the number of substitutions generated has very little impact on overall performance contrary to the ordering heuristic.
The substitution generation phase occurs once all type arguments have been generated. The bound limiting the number of monomorphising substitutions per formula is directly related to the ordering which dictates how such monomorphising substitutions are generated. The substitution ordering heuristic has a significant impact on monomorphising substitution generation. The ``age'' ordering of substitution orders substitutions generated in earlier iterations are ordered first. The ``random'' ordering randomly shuffles substitutions. Finally, the ``separation'' ordering separates the substitutions into groups of substitutions generated in the same iteration and generates monomorphising substitutions from each of these groups independently. Table \ref{subst_gen_table} shows that the values of bounds limiting the number of monomorphising substitutions only seems to affect performance when the ``separation'' heuristic is used.
% TODO find a better term than "separation heuristic" maybe separation ?
\begin{table}[th]
\caption{Evaluation of bounds for substitution generation}
\centering\begin{tabular}{@{}l*{3}{>{\centering\arraybackslash}p{5em}}@{}}
   \toprule
   & \multicolumn{3}{c}{substitution ordering} \\
   \multirow{1}{5em}{mono subst} & age & random & separation\\
   \cmidrule(lr){2-2}\cmidrule(lr){3-3}\cmidrule(l){4-4}
   2  & 161 & 178 & 175 \\
   5  & 161 & 178 & 180 \\
   7  & 161 & 178 & 182 \\
   10 & 161 & 178 &\bf{184} \\
   \bottomrule
\end{tabular}
\label{subst_gen_table}
\end{table}


Table \ref{pb_size} groups together the bounds related to the size of the output problem to be passed to the E prover. The absolute cap is the maximum number of formulae passed to E. The multiplier limits the total number of newly generated formulae based on the problem's initial number of formulae if it is bigger than the absolute cap. The performance of the E prover depends on the number of formulae it is given. To account for this, we test the effects of the amount of time allocated to E on the monomorphised problem against the size of the monomorphised problem. We observe that performance seems to be similar across configurations where the size of the output problem is sufficiently constrained. The effect of the the E timeout seems to be small beyond a minimal thershold.


\begin{table}[th]
\caption{Evaluation of bounds directly related to the size of the output problem}
\centering\begin{tabular}{@{}l*{12}{>{\centering\arraybackslash}p{1.5em}}@{}}
   \toprule
   & &&&& \multicolumn{4}{c}{formula cap} \\
   & \multicolumn{4}{c}{500} & \multicolumn{4}{c}{2000} & \multicolumn{4}{c}{\(\infty\)}\\
   \cmidrule(l){2-13}
   & &&&& \multicolumn{4}{c}{formula multiplier}\\
   \multirow{1}{5.5em}{E timeout (s)} & 1 & 2 & 3 & \(\infty\)& 1 & 2 & 3 & \(\infty\)& 1 & 2 & 3 & \(\infty\)\\
   \cmidrule(lr){2-5}\cmidrule(lr){6-9}\cmidrule(l){10-13}
   2    & 178 & 178 & 179 & 179 & 178 & 178 & 179 & 178 & 169 & 176 & 177 & 149 \\
   5 &\bf{186}& 186 & 186 & 185 & 186 & 186 & 186 & 186 & 173 & 182 & 185 & 149 \\
   10   & 186 & 186 & 186 & 185 & 186 & 186 & 186 & 186 & 173 & 182 & 185 & 148 \\
   30   & 184 & 184 & 184 & 183 & 184 & 184 & 184 & 184 & 168 & 178 & 183 & 125 \\
   \bottomrule
\end{tabular}
\label{pb_size}
\end{table}

For larger problems, the monomorphisation algorithm may timeout despite the bounds. In these cases neither Zipperposition nor E will have had a chance to attempt to solve the problem. To avoid this, a timer can interrupt the monomorphisation algorithm, after which Zipperposition resumes normal operation. Table \ref{mono_time} tests the amount of time that monomorphisation is allowed to run for against the number of iteration of the monomorphisation algorithm. The number of iterations and timeout seem to have almost no effect on the performance of the monomorphisation algorithm. % TODO numbers are not definitive

\begin{table}[th]
\caption{Evaluation of parameters related to the depth of monomorphisation}
\centering\begin{tabular}{@{}l*{4}{>{\centering\arraybackslash}p{1.5em}}@{}}
   \toprule
   & \multicolumn{4}{c}{mono time} \\
   \multirow{1}{4em}{loop nb} & 5 & 10 & 20 & 30\\
   \midrule
   2     & 191 & 191&\bf{192}&190\\
   3     & 191 & 191 & 191 & 190 \\
   4     & 190 & 190 & 190 & 189 \\
   5     & 190 & 190 & 190 & 189 \\
   6     & 189 & 190 & 189 & 189 \\
   \bottomrule
\end{tabular}
\label{mono_time}
\end{table}


\subsection{E as Zipperposition backend}

We conducted two evaluations of Zipperpostion with the monomorphisation algorithm and E as a backend. The first was on the 500 problems we used for option optimisation and the second on the remaining 1034 problems. This version of Zipperposition was compared against the base Zipperposition prover. The base Zipperposition instance was run wiht the sequential portfolio mode \verb|portfolio.sequential.py|. The results of these evaluation seem to suggest that some amount of overfitting did take place during the option optimisation phase because the number of problems solved only by the base version of Zipperposition grows disproportionately relative to the total number of problems.

Note that Zipperposition's portfolio mode seems to consider timeouts to be more of a polite suggestion than a hard limit, we consequently added a manual timeout of 30 seconds.

\begin{table}[ht]
\caption{Evaluation of Zipperposition without E vs. with E}
\centering\begin{tabular}{@{}lccc@{}}
   \toprule
   & without E & with E & Union \\
   \midrule
   500 problems benchmark   & 157 & 213 & 219 (-6 +62) \\
   fresh problems benchmark & 341 & 442 & 467 (-25 +126)\\
   \bottomrule
\end{tabular}
\end{table}


\subsection{Monomorphisation as preprocessor}

To evaluate the usefulness of iterative monomorphisation as an alternative to native polymorphism, several competitive higher-order polymorphic provers were run on the set of TPTP test problems. Evaluation of the monomorphisation approach is done in two steps, first each problem is monomorphised using the [insert configuration] configurations [options?]. Secondly, each prover is ran on the monomorphised problem and the results are tallied in the ``Mono'' column of the table. In addition to the polymorphic provers used in the ``Native'' tests, two monomorphic provers are run on the monomorphised problems. Note that instead of running each prover 30 seconds on the monomorphised problems, the monomorphisation time (rounded up to the nearest second) is subtracted to compare fairly against the ``Native'' column which does not have a similar preprocessing phase.

The ``Union'' column, adds up the total number of problems solved by both tests. It serves as an indicator of the usefulness of monomorphisation as a tool in a portfolio configuration.

\begin{table}[ht]
\caption{Evaluation of native polymorphism vs.\ monomorphisation}
\centering\begin{tabular}{@{}lccc@{}}
   \toprule
   & Native & Mono & Union \\
   \midrule
   E  &   & 0 & 0 \\
   Leo-III &  157 with E & 0 & 0 \\ 
   %Leo-III &  68 with cvc4 & 0 & 0 \\ % same as without anything for the native case
   Satallax &  & 0 & 0 \\
   %Vampire & 0 & 0 & 0 \\ % Vampire's parser fails on too many polymorphic problems (about 1/3 of them)
   Zipperposition & 339 & 0 & 0 \\[1.5\jot]
   Total & 0 & 0 & 0 \\
   \bottomrule
\end{tabular}
\end{table}


\break

% target page number: 0.5-1
%\section{Related work}
%\label{sec:related-work}
%
%  * monomorphisation:
%    Isabelle for \textit{smt} tactic (Böhme), \textit{metis}, Sledgehammer
%  * a precursor for this work is the \verb|POLY_ASSUME_TAC| preprocessor of
%    MESON for HOL Light. It is unfortunately not documented beyond the comment
%    ``Push duplicated copies of poly theorems to match existing assumptions''
%    but it appears to perform at least partial monomorphisation.
%
%  * alternative: encode polymorphism
%    * many approaches to encode monomorphic and polymorphic types, starting
%      with Enderton \cite{xxx}, Stickel \cite{xxx}, and Wick and McCune
%      \cite{xxx}
%    * Blanchette et al.~\cite{xxx} exploit ``monotonicity'' to erase type
%      information
%    * see Blanchette et al.~\cite[Section~9]{xxx} for a more detailed coverage
%      of encodings of polymorphism
%    
%%\li{https://www.tcs.ifi.lmu.de/mitarbeiter/jasmin-blanchette/enc_types_article.pdf}

% target page number: 0.5-1
\section{Conclusion}
\label{sec:conclusion}

We described a procedure for iteratively instantiating polymorphic types to
produce monomorphic problems as a preprocessor. Our primary motivation was to
improve the success rate of Zipperposition and its monomorphic E backend, and
indeed our evaluation shows a clear improvement. We also saw that even with
automatic provers that support polymorphism, iterative monomorphisation is a
better alternative in practice.

We see the following avenues for future work. First, iterative monomorphisation
blindly enumerates candidate instantiations, without exploiting any knowledge
about the logical structure of the formulas in which symbols occur. For
example, a lemma $\sym{p}\langle\alpha\rangle$ cannot be used to prove the
conjecture $\lnot \sym{p}\langle\ty{nat}\rangle$ because of the incompatible
polarities, but our procedure instantiates $\alpha$ with $\ty{nat}$ regardless.
Second, some automatic provers as well as tools like Sledgehammer include
a relevance filter that heuristically select a subset of the available axioms;
filters such as SInE \cite{xxx} and MePo \cite{xxx} are iterative and could be
interleaved with monomorphisation. Third, although one would expect native
implementations of polymorphism to outperform any preprocessor, our evaluation
clearly shows that this is not the case, therefore indicating that there is
considerable room for improvement on the native front.

\section*{Acknowledgements}

We thank Sascha Böhme for fruitful discussions.

This research is co-funded by the European Union (ERC, Nekoka, 101083038).
Views and opinions expressed are however those of the authors only and do not
necessarily reflect those of the European Union or the European Research
Council. Neither the European Union nor the granting authority can be held
responsible for them.

\bibliography{citations}

\end{document}
